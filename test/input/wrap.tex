% wrapped.tex - illustrating how to specify the size of the table
% columns.

% Andrew Roberts - September 2003

\documentclass[english]{article} 

%The Babel package adds multi-lingual support to Latex.  In particular,
%text that Latex automatically inserts, like dates, or section headings
%must be configured to the native language of the document.  The default
%is American. 

%The reason for adding babel in this instance seems silly, considering
%it is essentially the same language.  However, it also dictates the
%formatting of dates, which is different between us and the US. Adding
%the optional argument 'english' in the documentclass command, as well
%as invoking the babel package will recitfy the issue.
\usepackage{babel}
\usepackage{times}

\begin{document}

\title{Latex Tutorials 4 (Tables) Examples \\Taking control of your tables}
\author{Andrew Roberts}
\maketitle

Without specifying width for last column:

\begin{center}
	\begin{tabular}{ | l | l | l | l |}
	\hline
	Day & Min Temp & Max Temp & Summary \\ \hline
	Monday & 11C & 22C & A clear day with lots of sunshine.  However, the strong breeze will bring down the temperatures. \\ \hline
	Tuesday & 9C & 19C & Cloudy with rain, across many northern regions. Clear spells across most of Scotland and Northern Ireland, but rain reaching the far northwest. \\ \hline
	Wednesday & 10C & 21C & Rain will still linger for the morning. Conditions will improve by early afternoon and continue throughout the evening. \\
	\hline
	\end{tabular}
\end{center}

With width specified:

\begin{center}
	\begin{tabular}{ | l | l | l | p{5cm} |}
	\hline
	Day & Min Temp & Max Temp & Summary \\ \hline
	Monday & 11C & 22C & A clear day with lots of sunshine.  However, the strong breeze will bring down the temperatures. \\ \hline
	Tuesday & 9C & 19C & Cloudy with rain, across many northern regions. Clear spells across most of Scotland and Northern Ireland, but rain reaching the far northwest. \\ \hline
	Wednesday & 10C & 21C & Rain will still linger for the morning. Conditions will improve by early afternoon and continue throughout the evening. \\
	\hline
	\end{tabular}
\end{center}


\end{document}